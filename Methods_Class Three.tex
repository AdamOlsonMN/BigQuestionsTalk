\documentclass{beamer}
%
% Choose how your presentation looks.
%
% For more themes, color themes and font themes, see:
% http://deic.uab.es/~iblanes/beamer_gallery/index_by_theme.html
%
\mode<presentation>
{
  \usetheme{Boadilla}      % or try Darmstadt, Madrid, Warsaw, ...
  \usecolortheme{dove} % or try albatross, beaver, crane, ...
  \usefonttheme{default}  % or try serif, structurebold, ...
  \setbeamertemplate{navigation symbols}{}
  \setbeamertemplate{caption}[numbered]
} 

\usepackage[english]{babel}
\usepackage[utf8x]{inputenc}
\usepackage{booktabs}
\usepackage{multirow}
\usepackage{bigstrut}
\usepackage{setspace}
\usepackage{indentfirst}
\usepackage{tabulary}
\usepackage{prettyref}
\usepackage[flushleft]{threeparttable}
\usepackage{tikz}
\def\checkmark{\tikz\fill[scale=0.4](0,.35) -- (.25,0) -- (1,.7) -- (.25,.15) -- cycle;}
\setbeamertemplate{itemize items}[default]
\setbeamertemplate{enumerate items}[default]

\title[Empirical Approach]{POLS 2001: Political Science Research Methods: \\ The Empirical Approach in Political Science}
\author[Adam Olson]{Adam Olson}
%\institute[UMN]{University of Minnesota, Twin Cities}
\date{August 29th, 2016}

\begin{document}
\begin{frame}
	\titlepage
\end{frame}
\begin{frame}{Plan for the Week}
\begin{description}
\item [\textbf{Monday:}] Empiricism, parts of the Scientific Method, Question Assignment
\item [\textbf{Wednesday:}] Theory, `Empirical Research Process,' Workbook Exercise 
\item [\textbf{Friday:}] Concrete Parts of Research Process and Criticisms of the Empirical Approach
\end{description}

\end{frame}
\begin{frame}{Plan of the Day}
\tableofcontents
\end{frame}


\section{Empiricism}
\begin{frame}{Empiricism: Types of Claims}
\begin{description}
\item [Empirical:] A verifiable assertion of `what is'
\item [Normative:] An assertion of `what should be'
\item [Rhetorical:] Stating a belief like a it is fact.
\end{description}
\end{frame}

\begin{frame}{Empiricism: Parts}
\begin{itemize}
\item Empiricism uses observation to judge the tenability of arguments.
\item Empirical statements 
\begin{itemize}
\item can be believed and accepted to the extent that they are derived from empirical or observational evidence. 
\item does not depends on belief, authority
\end{itemize}
\item Non Empirical Statements Strategies - Friday
\end{itemize}
\end{frame}

\section{The Scientific Approach}
\begin{frame}{Elements of Empirical Claims and Knowledge}
\begin{itemize}
\item The scientific method:
\begin{itemize}
\item Findings are based on objective, systematic observation and verified through public inspection of methods and results.
\end{itemize}
\item The ultimate goal is to use verifiable results to construct causal theories that explain why phenomena behave the way they do.
\end{itemize} 
\textbf{Attempts to Implement Empiricism in an applied way}
\end{frame}
\begin{frame}{The Scientific Approach: Traits}
\begin{enumerate}
\item Empirical verification
\item Falsifiability
\item Non-normative research
\item Transmissible
\item Cumulative enterprise
\item Empirical generalization
\item Explanatory
\item Prediction
\item Probabilistic explanation
\item Parsimony
\end{enumerate}
\end{frame}
\subsection{Empirical Verification}
\begin{frame}{Empirical Verification}
Scientific Knowledge needs to be `known' or \textit{verifiable}.
\begin{itemize}
\item This is basic -- first steps
\item Cannot accept or reject claims if we cannot observe them in some way
\item Verification is not dependent on instinct or anything other than observation
\item `Science is the belief in the ignorance of experts'
\end{itemize}
At its core, empirical strategies are extremely critical of existing claims
\end{frame}
\subsection{Falsifiability}
\begin{frame}{Falsifiability}
A scientific claim needs to be \textit{falsifiable} or \textit{conditional}.
\begin{itemize}
\item Something isn't necessarily `proven' but it has merely not been disproven
\item The trick is to constantly seek to refute claims
\item A claim that does not allow the possibility of refutation cannot be empirical
\item `The electron is a theory we use; it is so useful in understanding the way nature works that we can almost call it real.'
\end{itemize}
A lot of people have made significant scientific careers going after the `conventional wisdom.'
\end{frame}
\subsection{Non-normative Research}
\begin{frame}{Non-normative Research}
A scientific claim needs to be \textit{non-normative.}
\begin{itemize}
\item Science does not make judgments as to whether or not something is good or bad.
\item However, the implications for the scientific conclusions can and are used that way 
\item Is all research non-normative? Are people who study mass incarceration or human rights abuses merely tying to uncover deeper truths?
\end{itemize}
\end{frame}
\subsection{Transmissible}
\begin{frame}{Transmissible}
Scientific knowledge must be transparent or \textit{`transmissible.'}
\begin{itemize}
\item Methods should be open
\item Data collection process should be described
\item Remember, no room for `trust.'
\item Allows for falsifiable and independent verification
\end{itemize}
\end{frame}
\subsection{Cumulative Enterprise}
\begin{frame}{Cumulative Enterprise}
Scientific claims should build on other scientific knowledge.
\begin{itemize}
\item It is important to constantly make human knowledge better rather than continually reinventing the wheel
\item \href{http://matt.might.net/articles/phd-school-in-pictures/}{\beamergotobutton{Click Me!}}
\end{itemize}

\end{frame}
\subsection{Empirical Generalization}
\begin{frame}{Empirical Generalization}
Scientific claims should seek to generalize to something bigger
\begin{itemize}
\item When we survey 1000 people, our interest in their opinions are instrumental so they may tell us about the country's opinions
\item Examining how specific species of animals evolve is supposed to allow insight into most animals
\end{itemize}
\end{frame}
\subsection{Explanatory}

\begin{frame}{Explanatory}
Scientific claims should seek to explain something
\begin{itemize}
\item $X$ is associated with $Y$?
\item Why is $X$ associated with $Y$?
\item How is $X$ associated with $Y$?
\end{itemize}
\end{frame}

\begin{frame}{Causality}
Some scientific claims are causal.
\begin{itemize}
\item $X$ causes $Y$
\item Spurious Causality
\item http://tylervigen.com/spurious-correlations

\end{itemize}
\end{frame}

\begin{frame}{Relevant Comic about Causality}
\begin{center}
\includegraphics[width=\textwidth]{correlation.png}
\end{center}
\end{frame}
\subsection{Prediction}
\begin{frame}{Prediction}
Some causal claims can be used to predict things
\begin{itemize}
\item Can be simple or complex
\item We might predict that someone who identifies as a Democrat will vote for Hillary Clinton
\item Remember Critical Mass? If we have more of a group of people in a legislature, we might predict that they will have more success at advancing bills relevant to their interests
\end{itemize}
\end{frame}
\subsection{Probabilistic Explanation}
\begin{frame}{Probabilistic Explanation}
Scientific claims are probabilistic in nature.
\begin{itemize}
\item Most things, especially in the social sciences are not deterministic
\item If X then Y
\item If X then probably Y
\item If X then sometimes Y
\end{itemize}
\end{frame}
\subsection{Parsimony}
\begin{frame}{Parsimony}
The best scientific claims are often the simplest
\begin{itemize}
\item Ockham's Razor
\end{itemize}
\end{frame}
\section{Wrapping Up}
\begin{frame}{Wrapping Up}
\begin{itemize}
\item Important to assess your own questions against these traits
\end{itemize}
\end{frame}
\end{document}